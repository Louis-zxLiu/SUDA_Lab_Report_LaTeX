\documentclass[a4paper,zihao=5]{ctexart} % 五号字体
% 页面设置
\usepackage[left=2.5cm,right=2.5cm,top=2.5cm,bottom=2.5cm]{geometry}

% 字体设置
% ctexart 默认会尝试自动检测系统字体。
% 如果需要强制指定字体,可以使用 fontset 选项或如下手动设置。

\setCJKmainfont[AutoFakeBold=true, AutoFakeSlant=true]{SimSun}
\setCJKsansfont{SimHei}
\setCJKmonofont{SimSun}
\setCJKfamilyfont{kaiti}{KaiTi}
\newcommand{\kaiti}{\CJKfamily{kaiti}}

% 其他宏包
\usepackage{fancyhdr}  % 页眉页脚
\usepackage{lastpage}  % 最后一页
\usepackage{tabularx}  % 表格
\usepackage{multirow}  % 多行表格
\usepackage{titlesec}  % 标题格式
\usepackage{ulem}      % 下划线
\usepackage{indentfirst} % 首行缩进
\usepackage{graphicx}    % 图片支持
\usepackage{listings}    % 代码块支持
\usepackage{xcolor}      % 颜色支持
\usepackage{caption}     % 图片标题支持
\usepackage{enumitem}    % 列表间距控制

% 代码块设置
\lstset{
    basicstyle=\ttfamily\footnotesize, % 字体大小
    keywordstyle=\bfseries\color{blue}, % 关键字颜色
    commentstyle=\itshape\color{gray},  % 注释颜色
    stringstyle=\color{red!60!black},   % 字符串颜色
    breaklines=true,       % 自动换行
    frame=single,          % 代码框
    numbers=left,          % 行号位置
    numberstyle=\tiny,     % 行号字体
    tabsize=4,             % 制表符宽度
    captionpos=b,          % 标题位置
    keepspaces=true,       % 保留空格
    showstringspaces=false % 不显示字符串中的空格
}

% --------------------------------------------------
% 变量定义 - 请在此处填写你的信息
% --------------------------------------------------
\newcommand{\school}{计算机科学与技术学院}
\newcommand{\class}{2022级 计算机科学与技术}
\newcommand{\myname}{张三}
\newcommand{\studentid}{2200000000}
\newcommand{\coursename}{操作系统实验}
\newcommand{\experimentno}{1}
\newcommand{\reporttitle}{实验报告正文}
\newcommand{\teacher}{李四}
\newcommand{\partner}{无}
\newcommand{\dateyear}{2025}
\newcommand{\datemonth}{12}
\newcommand{\dateday}{17}
% --------------------------------------------------

% 页眉页脚设置
\pagestyle{fancy}
\fancyhf{} % 清空当前设置
\renewcommand{\headrulewidth}{0pt} % 去掉页眉线
\fancyfoot[C]{第 \thepage 页,共 \pageref{LastPage} 页}
\fancyfoot[R]{\kaiti 教务处制}

% 标题格式设置
% 一级标题:一、  (小四号 12pt) 
\titleformat{\section}{\bfseries\fontsize{10.5pt}{12.6pt}\selectfont}{}{0em}{\thesection、}
% 二级标题:1.
\titleformat{\subsection}{\bfseries\fontsize{10.5pt}{12.6pt}\selectfont}{}{0em}{\thesubsection.}
% 标题编号
\renewcommand{\thesection}{\chinese{section}}
\renewcommand{\thesubsection}{\arabic{subsection}}

% 正文排版
\setlength{\parindent}{1em}
\setlength{\parskip}{0pt}
\titlespacing*{\section}{0pt}{0pt}{0pt}
\titlespacing*{\subsection}{0pt}{0pt}{0pt}
\setlist{topsep=0pt,partopsep=0pt,parsep=0pt,itemsep=0pt}

\begin{document}

% 标题
\begin{center}
    {\kaiti \fontsize{24pt}{24pt}\selectfont \uuline{苏州大学实验报告}}
\end{center}
\vspace{0.5em}

% 信息表格
\begin{center}
\kaiti \fontsize{12pt}{15pt}\selectfont
\renewcommand{\arraystretch}{1.5}
\setlength{\tabcolsep}{2pt}
% 设置表格列垂直居中
\renewcommand{\tabularxcolumn}[1]{m{#1}}
% 定义居中的 X 列,调整 hsize
% 注意:所有 X 列的 hsize 之和必须等于 X 列的数量(这里是 2)
\newcolumntype{Y}{>{\centering\arraybackslash\hsize=0.9\hsize}X}
\newcolumntype{Z}{>{\centering\arraybackslash\hsize=1.1\hsize}X}

\begin{tabularx}{\textwidth}{|p{4.5em}<{\centering}|Y|p{5.5em}<{\centering}|Z|p{3em}<{\centering}|p{4.5em}<{\centering}|p{3em}<{\centering}|p{6.5em}<{\centering}|}
    \hline
    院、系 & \school & 
    年级专业 & \class & 
    姓名 & \myname & 
    学号 & \studentid \\
    \hline
    课程名称 & \multicolumn{5}{c|}{\coursename} & 成绩 & \\
    \hline
    指导教师 & \teacher & 同组实验者 & \multicolumn{2}{c|}{\partner} & 实验日期 & \multicolumn{2}{c|}{\dateyear-\datemonth-\dateday} \\
    \hline
\end{tabularx}
\end{center}

\vspace{0.5em}

% 正文标题
\begin{center}
    \zihao{3} \heiti \coursename 实验 \underline{\makebox[2cm][c]{\experimentno}}:\underline{\makebox[6cm][c]{\reporttitle}}
\end{center}
\vspace{0.5em}

\section{实验目的}
\begin{enumerate}
    \item 掌握......
    \item 理解......
\end{enumerate}

\section{实验内容}
本次实验的主要内容包括......

\section{实验步骤和结果}
\subsection{步骤一}
描述步骤一的过程......

\subsection{步骤二}
描述步骤二的过程......

\section{实验总结}
通过本次实验,我学习到了......

\clearpage
\appendix
\begin{center}
    {\bfseries \fontsize{14pt}{16.8pt}\selectfont 附录:源代码}
\end{center}
\vspace{0.5em}
\noindent 1、实验环境:Visual Studio Code, GNU C++ 编译器(g++)

\noindent 2、(1) test.cpp
\begin{lstlisting}[language=C++]
#include <iostream>

int main() {
    // 打印 Hello World
    std::cout << "Hello, World!" << std::endl;
    return 0;
}
\end{lstlisting}

% --------------------------------------------------
% 图片插入模板 (请取消注释使用)
% --------------------------------------------------
% \begin{figure}[htbp]
%     \centering
%     % 请确保图片文件在同一目录下,或指定路径
%     % width=0.8\textwidth 表示图片宽度为文本宽度的 80%
%     \includegraphics[width=0.8\textwidth]{preview.jpg}
%     \caption{图片标题}
%     \label{fig:example}
% \end{figure}
% 交叉引用示例:如图 \ref{fig:example} 所示。
% --------------------------------------------------

\end{document}
